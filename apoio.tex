https://www.ime.usp.br/~pf/algoritmos/index.html#C-language
https://www.youtube.com/watch?v=oZeezrNHxVo&list=PLIfZMtpPYFP5qaS2RFQxcNVkmJLGQwyKE
 
- Ler variaveis Strings
https://www.cprogressivo.net/2013/03/Lendro-e-Escrevendo-Strings-em-C.html

https://www.estantevirtual.com.br/jeitodeler/harry-farrer-e-outros-algoritmos-estruturados-616612950

Arquitetura do Computador
http://producao.virtual.ufpb.br/books/camyle/introducao-a-computacao-livro/livro/livro.chunked/ch04s01.html

Hierarquias de Memórias
http://www.di.ufpb.br/raimundo/Hierarquia/Hierarquia.html

Controlador
http://www.eng.uerj.br/~ldmm/arquitetura/Conceitos_de_entrada_e_saida.pdf

1. Material Complementar
1.1. http://eletrica.ufpr.br/~rogerio/visualg/Help/linguagem.htm

2. Artigo Python e vantagens
2.1. https://medium.com/@fmasanori/hoje-sou-um-professor-feliz-python-no-ensino-de-programa%C3%A7%C3%A3o-26a92ba73dfb

3. Cursos Gratuítos Python e Diversos
3.1. https://courses.cognitiveclass.ai/
3.2. Cloud IBM
     https://console.bluemix.net/dashboard/apps/
3.3. https://pplware.sapo.pt/tutoriais/aprenda-dar-uns-toques-na-linguagem-programacao-python/
3.4. http://www.algoritmosempython.com.br/

4. Cursos Gratuítos Linguagem C
4.1. Curso de Linguagem C
4.2. https://www.inf.pucrs.br/~pinho/LaproI/IntroC/IntroC.htm

5. Programação onLine
5.1. https://ideone.com/
5.2. https://www.onlinegdb.com/   (http://question.onlinegdb.com/)
5.3. O melhor editor e interpretador de Algoritmos é do Brasil
     http://visualg3.com.br/
5.4. http://www.codeblocks.org/

6. Dicionário Inglês
6.1. https://www.linguee.com/

http://www.ifba.edu.br/professores/antoniocarlos/aula6ads.pdf

https://www.tecmundo.com.br/supercomputadores/58611-computadores-mainframes-decada-1980-falta-imagens.htm

*******************************************************************
Turmas Área - Turmas de Seg, Ter e Qui - Algoritmos Computacionais

-- Período de Avaliação
1. Período de Avaliação AP1: 08 a 11 de Out de 2018 (04 a 10/Out)
2. Período de Avaliação AP2: 03 a 06 de Dez de 2018 (01 a 07/Dez)
3. Período de Avaliação AP3: 10 a 13 de Dez de 2018 (10 a 15/Dez)
   E talvez Substitutiva AP1 e/oi Ap2
3. Período de Avaliação Exame Final e Substs: 17 a 22 de Dez de 2018

-- Período Upload AP1
1. Período Upload AP1: 11 a 18 Setembro de 2018
2. Período Upload AP2, Subst AP1 e AP2: 06 a 14 Novembro de 2018
ALG T1 01 AP2; 01 SUB AP1; 01 SUB AP2 - 03 PROVAS
ALG T2 01 AP2; 01 SUB AP1; 01 SUB AP2 - 03 PROVAS
CN     01 AP2; 01 SUB AP1; 01 SUB AP2 - 03 PROVAS

3. Período Upload AP3 Avulsa: 06 a 20 Novembro de 2018
4. Período Upload Exame Final : ??? de 2018

-- Lançamento de Notas na Caderneta Integrees
1. Período de Lançamento de Notas P1: De 11 a 19/10/2018
2. Período de Lançamento de Notas P2: De 07 a 11/12/2018
3. Período de Lançamento de Notas P3: De 23 a 26/12/2018

- Dúvidas Calendário Acadêmico
Upload AP3 Avulsa ???
Upload Exame Final ???
Aplicação Exame Final, Substitutiva: AP1, AP2, AP3 tudo no mesmo dia ?

-- Algoritmo Computacional - Avaliação AP1
Aula 08 - Avaliação P1 - 08/10/18 Turma Seg (Confirmado 08/10/18) 
Aula 08 - Avaliação P1 - 11/10/18 Turma Qui (Confirmado 11/10/18)

-- Cálculo Numérico
Aula 08 - Avaliação P1 - 09/10/18 Turma Ter (09/10/18)

1. Seg
1.1. Aula 4 - Parada: Slide 3 Estrutura Condicional

2. Qui
2.1. Aula 2 - Parada: Pg 60 expressões aritméticas ( Falta teste de mesa e exercícios)

*******************************************************************
Aula 00 - Slide Aula ?? - 06/08/18 e 09/08/18 (Não teve aula - Reposição)
Reposição 06/08/2018: 01/09/2018 as 09:00h Turma Algoritmo Computacional - Seg
Reposição 09/08/2018: 01/09/2018 as 09:00h Turma Algoritmo Computacional - Qui

Aula 01 - Slide Aula 1   - Introdução/Conceitos - 13/08/18
Aula 02 - Slide Aula 1/2 - Tipos de dados/Variáveis/Constantes - 20/08/18
Aula 03 - Slide Aula 2/4 - Expressões + Estrutura Condicional - 27/08/18
Aula 04 - Slide Aula 4/5.1 - Estrutura Repetição - 03/09/18
Aula 05 - Slide Aula 5.1 - Estrutura Repetição - 10/09/18
Aula 06 - Slide Aula 5.2 - Estrutura Repetição - 17/09/18
Aula 07 - Revisão + Exercícios - 24/09/18

Aula 08 - Avaliação P1 - 01/10/18 Turma Seg e 03/10/18 Turma Qui

Aula 09 - Correção P1 + Exercícios + Slide Aula 6 - 08/10/18
Aula 10 - Slide Aula 6/7 - 15/10/18
Aula 11 - Slide Aula 7 - 22/10/18
Aula 12 - Slide Aula 8 - 29/10/18
Aula 13 - Slide Aula 9 - 05/11/18
Aula 14 - Slide Aula 9 - 12/11/18

Aula 15 - Slide Complementar - Mod. Procedimentos + Exercícios - 19/11/18
Aula 16 - Slide Aula 3 - Modularização: Funções + Exercícios - 26/11/18

Aula 17 - Revisão + Exercícios - 03/12/18

Período Upload P2, Subst P1 e Subst P2: 06 a 14 Novembro de 2018
Período de Avaliação P2: 01 a 07 de Dezembro de 2018
Prazo de Lançamento de Notas P2: até 11/12/2018

Aula 18 - Avaliação P2 - 10/12/18
Aula 19 - 17/12/18 - Correção P2 + Exercícios

************************************************************
12. Modulação de algoritmos: contexto; conceitos; sintaxes; técnicas modulares; decomposição e
representação hierárquica (clareza e compreensão); refinamentos; manipulação; escopo de
variáveis; passagem de parâmetros.
13. Módulo tipo procedimento: contexto; conceitos; sintaxes; técnicas modulares; decomposição
e representação hierárquica (clareza e compreensão); refinamentos; manipulação; escopo de
variáveis; passagem de parâmetros.
14. Módulo tipo funções: contexto; conceitos; sintaxes; técnicas modulares; decomposição e repr

*********************

#include <stdio.h>
int main(void)
{ int tam;
  int a[tam], b[tam], aux;
  int cont = 0; int ind = 0;
 
  printf("Digite Tamanho: "); scanf("%d", &tam);
  
  for ( int x = 0; x < tam; x++)
    { printf("Digite Elemento [%d]: ", x); scanf("%d", &a[x]); 
    }
    
  for ( int x = 0; x < tam; x++){
    cont = 0;
    for ( int y = x + 1; y < tam; y++)
      if ( a[x] == a[y] )
      { cont++;
      }
     
	 if ( cont > 0 )
	 {  aux = 0;
	    for ( int w = 0; w <= ind; w++)
          if ( b[w] == a[x] )
            aux = 1;
	    if ( aux == 0 )
	    {	
          b[ind] = a[x];
          ind++;
		}
  	}
  }
  for ( int x = 0; x < ind; x++)
    printf("%d ", b[x]);

  printf("Total Repetidos: %d ", ind);
}