#include <stdio.h>

/*
int maior (int n, int v[])
{ 
   if (n == 1)
      return v[0];
   else {
      int x;
      x = maior (n-1, v);
      // x é o máximo de v[0..n-2] 
      if (x > v[n-1]) return x;
      else return v[n-1]; 
   }
}*/

int maior (int n, int v[]) {
   int x;
   if (n == 1) x = v[0];
   else {
      x = maior (n-1, v); 
      if (x < v[n-1]) x = v[n-1];
   }
   return x;
}

int main(void) 
	int a[4];
	
	for ( int x= 0; x<= 3; x++){
	printf("Digite Elemento"); scanf("%d", &a[x]);}
	
	printf("%d", maior(4, a));
}

// 1. Função de Fatorial com recursividade  
#include <stdio.h>
int fatorial ( int n) {
    int resultado;
    if ( n <= 1)
        resultado = 1;
    else
        resultado = n * fatorial( n - 1);
    return resultado;
}

int main()
{   int num, resultado;
 
    printf("Digite número \n");
    scanf("%d", &num);
 
    resultado = fatorial(num);
    printf("Fatorial %d = %d", num, resultado);

    return 0;
}

// 2. Ordenação de Vetor A - Solução 1

#include <stdio.h>
#define tamanho_vetor 4

int ordenaVetor(int *VetB, int tamanho_vet){
    int menor, i, y, x, cont;
    int VetAux[tamanho_vet];
    
    for (x = 0; x <= tamanho_vet-1; x++)
      VetAux[x] = VetB[x];
    
    for (x=0; x <= tamanho_vet-1; x++) {
        menor = VetAux[x]; cont = 0;
        for (y = 0; y <= tamanho_vet-1; y++) {
            if (menor < VetB[y]) {
                cont++;
            }
        }
        if (cont == 0)
          i = tamanho_vet - 1;
        else
          i = tamanho_vet - 1 - cont;
        
        VetB[i] = menor;
    }
    
}

int main()
{   int x;
    int VetA[tamanho_vetor];
    int VetC[tamanho_vetor];

    for (x = 0; x <= tamanho_vetor-1; x++){
        printf("Digite número %d = ", x + 1);
        scanf("%d", &VetA[x]);
        VetC[x] = VetA[x];
    }
    
    ordenaVetor(VetA, tamanho_vetor);
    
    printf("Vetor Original:");
    for (x = 0; x <= tamanho_vetor-1; x++){
        printf("%d ; ", VetC[x]);
    }

    printf("\nVetor Ordenado:");
    for (x = 0; x <= tamanho_vetor-1; x++){
        printf("%d ; ", VetA[x]);
    }

    return 0;
}

// 2. Ordenação de Vetor A - Solução 2

#include <stdio.h>
#define tamanho_vetor 5

    int ordenaVetor(int *VetB, int tamanho_vet){
    int aux, y, x;
    for (x=0; x <= tamanho_vet-1; x++) {
        for (y = x; y <= tamanho_vet-1; y++) {
            // Ordenação crescente   (VetB[x] > VetB[y])
            // Ordenação Decrescente (VetB[x] < VetB[y])
            if (VetB[x] > VetB[y]) {
                aux = VetB[x];
                VetB[x] = VetB[y];
                VetB[y] = aux;
            }
        }
    }
}

// Fatorial - Função Recursiva

int main()
{   int x;
    int VetA[tamanho_vetor];
    int VetC[tamanho_vetor];

    for (x = 0; x <= tamanho_vetor-1; x++){
        printf("Digite número %d = ", x + 1);
        scanf("%d", &VetA[x]);
        VetC[x] = VetA[x];
    }
    
    ordenaVetor(VetA, tamanho_vetor);
    
    printf("Vetor Original:");
    for (x = 0; x <= tamanho_vetor-1; x++){
        printf("%d ; ", VetC[x]);
    }
    
    printf("\nVetor Ordenado:");

    for (x = 0; x <= tamanho_vetor-1; x++){
        printf("%d ; ", VetA[x]);
    }

    return 0;
}

// Sequência de Fibonacci - Função Recursiva
#include <stdio.h>

int memo[] = {};
int fib(int n) {
    int f;
    //if ( n in memo ) return memo[n];
    //if ( n == memo[n] ) return memo[n];
    if (n<=2) return 1;
    else f = fib(n-1) + fib(n-2);
    memo[n] = f;
    return f;
}
// 1 + 1 + 2 + 3 + 5 + 8 + 13 + 21 + 34
int main()
{   int n, total;
    printf("Digite um Número? ");
    scanf("%d", &n);
    total = fib(n);
    printf("%d", total);

    return 0;
}

// Imprimir de 1 a 100 - Função Recursiva

int Imprime(int x){
    printf("%d ", 101 - x);
    if ( x == 1) return 1;
    return Imprime(x-1);
}
// 1 + 2 + 3 + 4 + 5 + 6 + 7 + 8 + 9 + 10

int main()
{  int x = 100;
    Imprime(x);
    return 0;
}

************ OK 
void fImprime(int num) {
    printf("%d ", 11 - num);
    if ( num != 1) 
      fImprime(num-1);
}

int main()
{   fImprime(10);
    return 0;
}

******** Fatorial

int fFatorial(int num) {
    if ( num == 0) return 1;
    else return num * fFatorial(num - 1);
}

int main()
{   int numero;
    scanf("%i",&numero);
    if ( numero >=0 )
	  printf("%d", fFatorial(numero));
	else
	  printf("Fatorial não existe");
    return 0;
}

*** Somatorio de 2 ate 100 dos pares
int fSomatorio(int num) {
    if ( num == 0) return 0;
	 return num + fSomatorio(num - 2);
}

int main()
{  	printf("%d", fSomatorio(10));
    return 0;
}
